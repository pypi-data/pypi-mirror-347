\documentclass[11pt]{article}
\usepackage{fullpage}

\title{The Meijer G-function\footnote
{Copyright 2007 by Edmond Orignac.
This file is released under the terms of the GNU General Public License, version 2.}}

\begin{document}

\maketitle

\section{Meijer G-function} 

The maxima function \texttt{meijer\_gred(x,b,a,c,d)} where 
\texttt{x} is an expression, and \texttt{b,a,c,d} are lists
attempts to express the Meijer G-function in terms of simpler
functions, namely generalized hypergeometric functions, bessel
functions and elementary transcendental functions. When no 
simplification can be performed, it 
returns \texttt{meijer\_g(x,b,a,c,d)}.  


\subsection{Some definitions}

The Meijer G-function is defined as a Mellin-Barnes integral\cite{meijer1951,braaksma1962,braaksma1975}:

\begin{equation}
  \label{eq:g-function-def}
  G^{mn}_{pq}\left(z,\begin{array}{c} a_1\ldots a_n;a_{n+1} \ldots a_p \\ b_1\ldots b_m;b_{m+1} \ldots b_q \end{array} \right) = \frac 1 {2i\pi} \int_C \frac{\prod_{k=1}^m \Gamma(b_k-s) \prod_{j=1}^n \Gamma(1-a_j+s)}{\prod_{k=m+1}^q \Gamma(1-b_k+s) \prod_{j=n+1}^p \Gamma(a_j-s)} z^s ds   
\end{equation}

Where the contour $C$ is closed in an appropriate way to ensure the convergence
of the integral. It is also required that no $a_j-b_k$ is an integer. 

Many integral transforms of special functions can be expressed in terms of Meijer G-functions.\cite{maloo1966,shah1972,bajpai1968,bajpai1974,dahiya1988}

The Meijer G-function has the following properties\cite{roach1997}: 
 
It is invariant under any permutation of the parameters $a_1,\dots,a_n$ among themselves, any permutation of the parameters $b_1,\ldots, b_m$ among themselves, any permutation of the parameters $a_{n+1},\ldots a_p$ among themselves, and 
any permutation of the parameters $b_{m+1},\ldots, b_q$ among themselves as 
a trivial consequence of its definition. Furthermore, one has:  

\begin{equation}\label{contin}
   G^{mn}_{pq}\left(z,\begin{array}{c} a_1\ldots a_n;a_{n+1} \ldots a_p \\ b_1\ldots b_m;b_{m+1} \ldots b_q \end{array} \right)= G^{nm}_{qp}\left(\frac 1 z,\begin{array}{c}1-b_1\ldots 1-b_m;1-b_{m+1} \ldots 1-b_q\\  1-a_1\ldots 1-a_n;1-a_{n+1} \ldots 1-a_p  \end{array} \right)
\end{equation}

\begin{equation} \label{shift}
  G^{mn}_{pq}\left(z,\begin{array}{c} a_1+\alpha\ldots a_n+\alpha;a_{n+1}+\alpha \ldots a_p+\alpha \\ b_1+\alpha\ldots b_m+\alpha;b_{m+1}+\alpha \ldots b_q+\alpha \end{array} \right)=z^\alpha  G^{mn}_{pq}\left(z,\begin{array}{c} a_1\ldots a_n;a_{n+1} \ldots a_p \\ b_1\ldots b_m;b_{m+1} \ldots b_q \end{array} \right)
\end{equation}

When no $b_j-b_k$ is an integer, the theorem of L.J. Slater allows the Meijer 
G-function to be transformed into a sum of ${}_pF_{q-1}$ 
generalized hypergeometric functions.\cite{roach1997} 

For $p<q$ or for $p=q$ and $|z|<1$ one has:

\begin{eqnarray}\label{slater}
&&G^{mn}_{pq}\left(z,\begin{array}{c} a_1\ldots a_n;a_{n+1} \ldots
       a_p \\ b_1\ldots b_m;b_{m+1} \ldots b_q \end{array} \right)= \\[6pt]
&&\sum_{k=1}^m  \frac{ z^{b_k} \prod_{j=1}^n \Gamma(1+b_k-a_j) \prod_{l=1\atop l\ne k}^m \Gamma(b_j-b_k)}{\prod_{j=n+1}^p \Gamma(a_j-b_k) \prod_{j=m+1}^q \Gamma(1+b_k-b_j)} {}_pF_{q-1} (\{1+b_k-a_j\};\{1+b_k-b_l\}_{l\ne k};(-1)^{m+n-p} z)\nonumber
\end{eqnarray}

The case of $q>p$ or $p=q$ and $|z|>1$ can be treated by applying first (\ref{contin}) and then (\ref{slater}). The condition is then that 
there are no $a_j-a_k$ that are integer.
  
\subsection{What the maxima script does}

When one enters \texttt{meijer\_gred(x,b,a,c,d)} with \texttt{a=[a$_1$,\ldots,a$_n$]}, \texttt{b=[b$_1$,...b$_m$]}, \texttt{c=[a$_{n+1}$,...,a$_p$]}, \texttt{d=[b$_{m+1}$,...,b$_q$]}, the maxima script attempts to find a simplified expression for:

\begin{equation}
  G^{mn}_{pq}\left(x,\begin{array}{c} a_1\ldots a_n;a_{n+1} \ldots a_p \\ b_1\ldots b_m;b_{m+1} \ldots b_q \end{array} \right)
\end{equation}
 
By applying first some simple identities,\cite{meijer1951,wille1988} and if that fails applying the Slater  formula (\ref{slater}) after checking that its condition of validity is met. 

One can see that the order in which the parameters have to be entered is the following: bottom left, then top left, then top right, then bottom right, i. e. blocks of parameters are read in the anticlockwise sense.


\subsection{Limitations} 

No numerical calculation or asymptotic expansion can be 
 performed by the script, although asymptotic expansions for Meijer G-functions are known.\cite{braaksma1962} 

The differentiation formulas for the G-functions are not used 
nor defined by the script.\cite{roach1997} This is (in principle) 
easy to add with \texttt{gradef} for differentiation with respect to $z$ 
but more difficult in case of differentiation with respect to the parameters. 
 

Relations between contiguous functions are not used by the script, but 
an algorithm exists to take advantage of these.\cite{roach1997}   
The relation (\ref{shift}) is not used in simplifications that are not
based on Slater's formula. 

Some identities for the G-function have been left out.\cite{meijer1932,meijer1935,meijer1939,meijer1951} Many more could be found in the Bateman Manuscript
project.
 


\section{MacRobert's E-function}

The MacRobert E-function\cite{macrobert1958,macrobert1959c,macrobert1959b,macrobert1959a,macrobert1959,macrobert1960,macrobert1960a,macrobert1961,macrobert1961a,macrobert1962,macrobert1962a,ragab1962,ragab1993,ragab1962a,ragab1954} is also defined by a Mellin-Barnes integral. 
This integral is a particular case of the integral defining 
Meijer's G-function, and one has:

\begin{equation}
  E(a_1\ldots a_p;b_1\ldots b_q;z)=G^{p1}{q+1,p}\left(z,\begin{array}{c} 1;b_1\ldots b_q\\ a_1\ldots a_p \end{array} \right)
\end{equation}

This identity is used by the function \texttt{macrobert\_ered(a,b,x)} to derive simplified forms of MacRobert's E-function.

 

%\bibliographystyle{acm} 
%\bibliography{maxima}

\begin{thebibliography}{10}

\bibitem{bajpai1968}
{\sc Bajpai, S.~D.}
\newblock A finite integrals involving a confluent hypergemoetric and a meijer
  g-function.
\newblock {\em Portugaliae Mathematica 28\/} (1968), 55.
\newblock Available online from National Library of Portugal.

\bibitem{bajpai1974}
{\sc Bajpai, S.~D.}
\newblock Expansion formula for the product of meijer g-function and bessel
  functions.
\newblock {\em Portugaliae Mathematica 33\/} (1974), 35.
\newblock Available online from National Library of Portugal.

\bibitem{braaksma1975}
{\sc Braaksma, B.}
\newblock In memoriam c. s. meijer.
\newblock {\em Nieuw Archief voor Wiskunde 23\/} (1975), 95.

\bibitem{braaksma1962}
{\sc Braaksma, B. L.~J.}
\newblock Asymptotic expansion and analytic continuation for a class of
  mellin-barnes integrals.
\newblock {\em Compositio Mathematicae 15\/} (1962--1964), 239.
\newblock available from \texttt{http://www.numdam.org}.

\bibitem{dahiya1988}
{\sc Dahiya, R., and Jowhar, I.}
\newblock Two dimensional laplace transforms of generalized hypergeometric
  functions.
\newblock {\em Internat. J. Math. \& Math. Sci. 2\/} (1988), 167.

\bibitem{macrobert1958}
{\sc MacRobert, T.}
\newblock Integrals involving e-functions.
\newblock {\em Mathematische Zeitschrift 69}, 234 (1958).
\newblock Available from texttt{http://www.digizeitschriften.de}.

\bibitem{macrobert1959c}
{\sc MacRobert, T.}
\newblock Infinite series of e-functions.
\newblock {\em Mathematische Zeitschrift 71\/} (1959), 143.
\newblock Available from \texttt{http://www.digizeitschriften.de}.

\bibitem{macrobert1959a}
{\sc MacRobert, T.}
\newblock Integrals of products of e-functions.
\newblock {\em Mathematische Annalen 137\/} (1959), 412.
\newblock Available from texttt{http://www.digizeitschriften.de}.

\bibitem{macrobert1959b}
{\sc MacRobert, T.}
\newblock The multiplication formula for the gamma function and e-function
  series.
\newblock {\em Mathematische Annalen 139\/} (1959--1960), 133.
\newblock Available from texttt{http://www.digizeitschriften.de}.

\bibitem{macrobert1960a}
{\sc MacRobert, T.}
\newblock Recurrence formulae for the e-functions.
\newblock {\em Mathematische Zeitschrift 73\/} (1960), 254.

\bibitem{macrobert1962a}
{\sc MacRobert, T.}
\newblock Barnes integrals as a sum of e-functions.
\newblock {\em Mathematische Annalen 147\/} (1962), 240.
\newblock Available from \texttt{http://www.digizeitschriften.de}.

\bibitem{macrobert1962}
{\sc MacRobert, T., and Ragab, F.}
\newblock E-function series whose sums are constants.
\newblock {\em Mathematische Zeitschrift 78\/} (1962), 231.

\bibitem{macrobert1959}
{\sc MacRobert, T.~M.}
\newblock product of e-functions.
\newblock {\em Pacific Journal of Mathematics ??\/} (1959), 759.
\newblock available from \texttt{http://www.projecteuclid.org}.

\bibitem{macrobert1960}
{\sc MacRobert, T.~M.}
\newblock product of e-functions.
\newblock {\em Pacific Journal of Mathematics ??\/} (1960), 309.
\newblock available from \texttt{http://www.projecteuclid.org}.

\bibitem{macrobert1961}
{\sc MacRobert, T.~M.}
\newblock Evaluation of an e-function when three of its upper parameters differ
  by integral values.
\newblock {\em Pacific Journal of Mathematics ??\/} (1961), 999.
\newblock available from \texttt{http://www.projecteuclid.org}.

\bibitem{macrobert1961a}
{\sc MacRobert, T.~M.}
\newblock Evaluation of an e-function when three of its upper parameters differ
  by integral values.
\newblock {\em Pacific Journal of Mathematics ??\/} (1961), 999.
\newblock available from \texttt{http://www.projecteuclid.org}.

\bibitem{maloo1966}
{\sc Maloo, H.}
\newblock Integrals involving products of bessel function and meijer's
  g-function.
\newblock {\em Monatshefte f\"ur Mathematik 70\/} (1966), 127.
\newblock available from \texttt{http://www.digizeitschriften.de}.

\bibitem{meijer1932}
{\sc Meijer, C.~S.}
\newblock Asymptotische entwicklungen von besselschen und hankelschen
  funktionen f\"ur grosse...
\newblock {\em Math. Ann. 108\/} (1932), 21.
\newblock available from \texttt{http://www.digizeitschriften.de}.

\bibitem{meijer1939}
{\sc Meijer, C.~S.}
\newblock Integraldarstellung fur struvesche und besselche funktionen.
\newblock {\em Compositio Mathematicae 6\/} (1939), 348.
\newblock available from \texttt{http://www.numdam.org}.

\bibitem{meijer1935}
{\sc Meijer, C.~S.}
\newblock Neue integraldarstellungen aus der theorie der whittakerschen und
  hankelschen funktionen.
\newblock {\em Mathematische Annalen 112\/} (1939), 469.
\newblock available from \texttt{http://www.digizeitschriften.de}.

\bibitem{meijer1951}
{\sc Meijer, C.~S.}
\newblock Neue integraldarstellung fur besselche funktionen.
\newblock {\em Compositio Mathematicae 8\/} (1951), 49.
\newblock available from \texttt{http://www.numdam.org}.

\bibitem{ragab1954}
{\sc Ragab, F.}
\newblock New integrals involving bessel-functions.
\newblock {\em Mathematische Zeitschrift 61\/} (1954), 386.
\newblock Available from \texttt{http://www.digizeitschriften.de}.

\bibitem{ragab1962}
{\sc Ragab, F.~M.}
\newblock Neue integraldarstellungen aus der theorie der macrobertschen und
  besselschen funktionen.
\newblock {\em Mathematische Zeitschrift 79\/} (1962), 147.

\bibitem{ragab1962a}
{\sc Ragab, F.~M.}
\newblock Integrals involving products of bessel functions.
\newblock {\em Mathematische Zeitschrift 80\/} (1962--1963), 177.

\bibitem{ragab1993}
{\sc Ragab, S.~F.}
\newblock Finite and infinite series involving macrobert's e-function.
\newblock {\em Portugaliae Mathematica 50\/} (1993), 507.
\newblock Available online from National Library of Portugal.

\bibitem{roach1997}
{\sc Roach, K.}
\newblock Meijer g-function representations.
\newblock In {\em ISSAC'97 proceedings\/} (New York, 1997), ACM.
\newblock Available from CITESEER.

\bibitem{shah1972}
{\sc Shah, M.}
\newblock On generalized meijer's and associated generlaized legendre
  functions.
\newblock {\em Portugaliae Mathematica 31\/} (1972), 55.
\newblock Available online from National Library of Portugal.

\bibitem{wille1988}
{\sc Wille, L.~T.}
\newblock A novel class of g-function integrals.
\newblock {\em Journal of Mathematical Physics 29\/} (1988), 599.

\end{thebibliography}

\end{document}
